\chapter{Background}

There are many unique challenges to building autonomous vehicles for underwater environments. And, several other university labs and organizations have build vehicles for aquatic exploration. This section will discuss these challenges and several of the most relevant vehicles. 

\section{Challenges of Underwater}
Working in underwater environments presents a number of challenges not experienced by aerial, ground, or surface vehicles. These include physical challenges, difficulty of communication, and lack of global positioning systems. 
\subsection{Physical Challenges}
First and foremost is the need to seal the vehicle for water and pressure. The uDrone will need to maintain a watertight seal down to 30m of water. This is made more challenging by the need to have some components exterior to the enclosure which houses the electronics, necessitating the creation of watertight pass-throughs. As the vehicle goes to deeper depths this becomes increasingly harder. Every 10m of water adds an additional atmosphere of pressure to the vehicle. This pressure causes two problems. First, enclosures must be strong enough to resist the force of the pressure change and not fail structurally. Second, o-rings and other seals shrink under pressure, which can lead to leaks. 

In order to achieve this, the uDrone uses a Blue Robotics enclosure to keep the electronics dry inside. In order to handle the cable pass-throughs, and o-ring sealed pass-through is attached to the rear of the enclosure and then the potting compound is used to seal the area between the cable and pass-through. 

An underwater vehicle also has to manage its buoyancy. If the vehicle is too heavy or too light it will end up moving through the water column at an angle, increasing drag and decreasing efficiency. Additionally, if the centers of buoyancy and mass do not line up then there will be a moment acting on the vehicle, complicating modeling and control. While this presents challenges, it can also lead to opportunities. By creating a vehicle that is slightly positively buoyant or by having a mechanism that allows a vehicle to drop ballast, a vehicle will float and can be recovered in the event of failure. Since the main camera on the uDrone is front-facing, the center of buoyancy can be set slightly behind the center of gravity. This will cause it to point down when hovering, but in motion, the thrusters can compensate for the moment. 

Many autonomous vehicles are able to simplify certain aspects of the model. For example, a vehicle moving through the air does not need to worry about added mass from the layer of air near the vehicle surface. In fact, air drag is often ignored unless a drone is moving at high speeds. This is not possible underwater due to the greater viscosity and density of water compared to air. In order to get reliable system modeling, expensive and complicated fluid dynamic software must be used. Since this is often impractical for applications such as the uDrone, the values for added mass and drag are often estimated and then verified experimentally. 

Similar to the effects of wind on an aerial vehicle, underwater vehicles must be able to compensate for currents. This is made more challenging as the speed of the vehicle cannot be verified by GPS. Instead, visual odometry must be used to determine the effects of current on the vehicle and aid in compensation. 

\subsection{Communication Challenges}
WiFi or radio signals do not penetrate to useful depths underwater. Radio waves in the MHz or higher frequency range typically penetrate less than 1 meter underwater. Since the radio controllers used on drones communicate at 900 MHz and WiFi uses 2.4 GHz or 5 GHz, none of these are viable communication options. 

Optimal communication is possible but is limited by line of sight and water column turbidity \parencite{light-coms}. Optimal communication has been implemented in a simulation for the uDrone, where it follows a boat that has a visual marker mounted underneath. This is reasonable for some use cases of the uDrone, but cannot be relied on for all scenarios.

% https://www.hydromea.com/underwater-wireless-communication/
The most common method of real-time communication with underwater vehicles is via a tether. This allows for fast and reliable connections but limits the movement of the vehicle based on cable length and cable management. While the uDrone is capable of functioning with a cable for testing purposes, it is designed to operate without one. 

The only reliable method of wireless underwater communication is via acoustic transponders. However, these devices tend to be large, expensive, and power-hungry. While one day it might be useful to have wireless acoustic communications at specific sites of uDrone deployment, it is not practical in the initial use case.

\subsection{Global Positioning and Navigation Challenges}
GPS satellites transmit data at high frequencies, in the GHz range. For this reason, traditional, satellite-based GPS is not practical for use underwater. The uDrone is still outfitted with GPS so it can determine its location on the surface, but it cannot rely on this as a method of localization while underwater.

It is possible to use acoustic signals to localize an underwater vehicle. This requires three or more surface-based acoustic broadcasters. An underwater vehicle can then triangulate between these sources in order to determine its position relative to the sources. This is basically the equivalent to creating an underwater GPS system. While it is possible that this could be implemented at some operational sites in the future, it is not practical to build in the beginning and is therefore not a reliable way for the uDrone to localize.

Due to the lack of GPS, the uDrone must use visual and inertial based odometry to determine its location. LIDAR, which is the measurement tool of choice of most aerial and land-based robots, is severely limited underwater for two reasons. First, lower frequency light is absorbed quickly underwater, and since most LIDARs use these lower frequencies, the range is severely limited. Second, murky water greatly limits the ability of light to pass through water, and since it is very common to have many dissolved particles in the water column this makes LIDAR inaccurate. Therefore, the best type of sensor to measure distance underwater is with an echo-sounder. 

Another tool typically used by land-based robotics to understand their surroundings is using depth cameras. There are two main types of depth cameras: stereo and infrared. Stereo cameras calculate the pixel depth of an image by looking at the difference in location between the images from two cameras placed a known distance apart. Infrared cameras work by projecting a pattern in infrared light onto surfaces and using the deformations to calculate distance. Since lower frequency lights, such as red and infrared, are absorbed most quickly by water, infrared depth cameras do not work well underwater. For this reason, a stereo camera is the most practical type of depth camera for use underwater. 

\section{Previous Work}
Existing Autonomous Underwater Vehicles (AUVs) and Remote Operate Vehicles (ROVs) tend to fall into 2 categories: Flight style and hover style. Flight style vehicles are similar to torpedoes. They typically have long, cylindrical bodies with a single thruster at the rear and control surfaces to control roll, pitch, and yaw while moving. These vehicles are typically used in the open ocean to complete larger area surveys, which necessitates their long range and higher speeds. The trade offs for speed, endurance, and range are less controllability, agility, and precision. Hover style vehicles have nearly opposite abilities. They are typically box shaped with multiple thrusters in various directions allowing for precise movement in any direction. Due to the challenge of controlling these highly maneuverable vehicles, they are typically ROVs, meaning there is a pilot on the surface connected via a tether and they are not autonomous. 

In this chapter I will present four vehicles, one each in the flight and hover style and then two in a newer category of underwater vehicle, called hydrobatic. Hydrobatic vehicles are both fast and agile, balancing the trade-offs between the range and speed or flight style vehicles and the maneuverability of hover style vehicles. The two hydrobatic vehicles presented here served as inspiration for the uDrone. This section is not meant to be a comprehensive review of underwater vehicles. Instead, it will serve to show some examples and give some background of the research that inspired the uDrone. 

\subsection{HippoCampus $\mu$UAV}
The HippoCampus micro underwater vehicle was developed at the Institute of Mechanics and Ocean Engineering, Hamburg University of Technology in Germany. This small, agile, and inexpensive vehicle was developed for use in swarms. Similar to the uDrone thruster configuration, the HippoCampus drew on inspiration from multi-rotor aerial vehicles to develop a four propeller design \parencite{hipp1}. In fact, the open-source code developed for this vehicle has been a huge help in designing gazebo simulations and PX4 controllers for the uDrone. Robust models were derived for the HippoCampus which were used to show its ability to function as a submerged Furuta Pendulum \parencite{hipp_pen}. Designed to be inexpensive and work in swarms, the HippoCampus has limited capacity for sensors. It uses a gyroscope, compass, and depth sensor, but does not make room for any visual or sonar sensors.

\subsection{AQUA: an Aquatic Walking Robot}

The AQUA robot was developed over 10 years ago by the  Mobile Robotics Lab at McGill University. It is based on the RHex hexapod robot. Its original intent was to create an amphibious robot that could transition from walking on land to swimming in water. Instead of thrusters, which are used by the majority of underwater vehicles, the AQUA robot uses six fins to propel itself through the water \parencite{aqua}. More recently, this platform has been used with an on board GPU in order to handle vision based navigation \parencite{MandersonGPU}.

The AQUA, along with its simulation environment, is a good example of an underwater robotic testbed leading to innovation. The work has transitioned from hardware to computer vision and autonomous navigation. For example, using this system a new vision based underwater navigation tool called Nav2Goal was created. In this, the underwater vehicle moves to a goal defined by the user. Along the way, it not only avoids obstacles but also takes paths that bring it near to areas of greater interest. This is particularly useful in coral reefs when a vehicle can decide to follow a path along a coral reef instead of a sandy bottom on its way to its goal \parencite{manderson2020visionbased}.

While this vehicle satisfies most of the requirements of the uDrone project, it was mainly developed as a research vehicle and for amphibious implementations, making it not ideally suited to the project needs. 

\subsection{Light Autonomous Underwater Vehicle}
The Light Autonomous Underwater Vehicle (LAUV) was developed in 2012 by the Laboratório de Sistemas e Tecnologia Subaquática (LSTS) at the Universidade do Porto in Protugal and is manufactured by OceanScan Marine Systems \& Technology. This vehicle was designed to be small enough to be carried by one person while still having full capabilities for scientific and defence surveys. It is built in the ‘flight’ or ‘torpedo’ style of underwater vehicle, having a long cylindrical body with a single thruster at the rear and control surfaces to control roll, pitch, and yaw while moving. The LAUV has an endurance of 6-8 hours at a speed of 1.4m/s, which would satisfy the same needs as the uDrone \parencite{lauv}. However, due to the flight style design it would not be able to reliably and precisely follow the terrain of the reef as necessitated by this project.

\subsection{Blue Robotics ROV2}
There are many ROVs exploiting the oceans, from the 2,400 kg ROV Minerva used for deep ocean research \parencite{minerva}, to the 3.4 kg Sofar Trident consumer underwater drone \parencite{sofar}. Because it uses many similar hardware components along with an open source software stack, the Blue Robotics ROV2 will be highlighted here. This vehicle is sold as a kit by Blue Robotics, utilizing many of its components. It is meant for both recreational enthusiasts and small scale research. The vehicle is from the surface via a tether. Onboard, it houses a PixHawk flight controller running ArduPilot, an alternative to PX4. Additionally, cameras, echo sounders, and other sensors can be added to the vehicle. It has a maximum speed of 1.5 m/s and a battery life of about 2 hours \parencite{bluerov}.

The Blue Robotics ROV2 has been used for autonomous underwater navigation experiments. One such set of experiments, conducted at the French-Mexican Laboratory on Computer Science and Control. The authors used the tether of the ROV2 to retrieve sensor data from underwater. This data was processed on a remote ground station and then sent back to the vehicle, again via the tether. The whole system operated with ROS and the MAVROS interface to ArduPilot. This allowed for real time and off board computer vision based autonomy \parencite{rtcv}.
