\chapter{Motivation}

\section{Science Need}

Greg currently dives with a DiveX Piranha Scooter, allowing him to travel at 1 knot (0.51 m/s) for about 4 hours. In order to stay under this long, he uses a rebreather. A rebreather is a SCUBA diving apparatus that filters out the exhaled carbon dioxide from the diver and augments it with stored oxygen. This allows for a significantly longer time spent underwater. As Greg travels over the reef, he keeps the sensor and light source between 10 and 30 cm away from the reef.

This data is used to calibrate satellite and Global Airborne Observatory (GAO). Satellite citation: The Sensitivity of Multi-spectral Satellite Sensors to Benthic Habitat Change
Cite interview with Greg Asner on October 12 or 13, 2019

\section{Dive Saftey}

The risk of death from scuba diving is low, with an estimated annual death rate of 2 deaths per 100,000 recreational scuba divers in the United States. Many of these deaths are due to cardiovascular health and diver errors (cite DAN 2016 Report). These risks are mitigated in the scientific diving community by the increased requirements set out by the American Association of Underwater Scientists (AAUS), which require additional training and medical checks (cite AAUS regulations). ASU and all associated labs follow these requirements for diving activities.

Dive injury, however, is significantly more common than fatality, with an estimated 15 occurrences per 10,000 dives. While there are many potential avenues for dive injury, including interactions with the marine environment, barotrauma, and gas contamination, one in particular can strike divers at any level and is particularly exacerbated by long and repeat dives. This is decompression sickness. Decompression sickness is the second most common type of dive injury, accounting for 27\% according, to the Divers Alert Network (DAN), the world's largest scuba diving safety association. (Barotrauma is the most common accounting for 41\% of injuries). 

Decompression sickness, or DCS, is caused when gas bubbles absorbed by body tissue at higher pressures underwater are not able to properly “off-gas” or reabsorb. This can cause bubbles to form in various regions of the body. Most commonly this occurs in joints or tissue which can impair motor function and have permanent effects. In some cases a bubble can form in the circulatory system, blocking blood flow, which can be fatal (DCS Paper).

In order to reduce the risk of DCS while diving, scuba divers use tools to monitor the dissolved gasses in their system. Originally, this was done with tables before and after a dive but most modern divers use a dive computer which tracks their gas absorption. This risk can also be mitigated for especially deep or long dives by mixing inert gasses into the breathable air in order to limit the amount of any one gas that gets absorbed. By using an autonomous underwater vehicle to conduct research, as opposed to human divers, these risks can be completely remediated. 

\section{Underwater Vehicle System}
While designed with these requirements in mind, the uDrone is useful beyond this specific implementation. It is quickly becoming a platform for innovation in autonomous underwater navigation within the DREAMS Lab at ASU. One thesis has already been written leveraging the system for multi-robot coordination and several others are in the works at this time.

This platform has several benefits for innovation:
\begin{itemize}
    \item An existing simulation environment allows for rapid software prototyping. Robotic software developers can focus on writing code, not building simulation environments, and configuring vehicles. 
    \item Transitioning from simulation to reality is made easier by leveraging this existing platform.
    \item The simulation can be continuously updated as more real-life trials are run with the vehicle. 
\end{itemize}
    
Additionally, building a vehicle from scratch is less expensive and easier to maintain than a purchased vehicle. This is because all the components are either off the shelf or built in house. This allows for the efficient acquisition of backup parts. The DREAMS lab team, having intimate knowledge of the vehicle, can troubleshoot and perform maintenance without outside help. Since the cost is low, it is possible to build multiple vehicles and perform swarm activities as well.

A good example of similar platform leading to innovation can be seen at Greg Dudek’s Lab at McGill. They AQUA robot was built over 10 years ago and students are still using it today for research. The work has transitioned from hardware to computer vision and autonomous navigation. For example, using this platform a new vision based underwater navigation tool called Nav2Goal was created. In this, the underwater vehicle moves to a goal defined by the user. Along the way, it not only avoids obstacles but also takes paths that bring it near to areas of greater interest. This is particularly useful in coral reefs when a vehicle can decide to follow a path along a coral reef instead of a sandy bottom on its way to its goal. (cite Autonomous Navigation for Unmanned Underwater Vehicles: Real-Time Experiments Using Computer Vision).

Later in this thesis, I present one such example of innovation and a controller implementation on the uDrone system.

% \section{Personal Motivation}
% My interest in building the uDrone with the DREAMS Lab at ASU started during my first semester as a Master’s Student in Robotics and Autonomous Systems. I took Prof. Das’ Exploration Systems Design class. My final project for this class was the construction of a boat to start the lab’s stable of aquatic vehicles. For this project, I handled all aspects of construction, system design, assembly, and configuration. This was my first intro to building an autonomous system from the ground up. The boat has been used for several projects in our lab and Prof. Das’ class. After successfully building and deploying this vehicle, and based on my personal experience with scuba diving, I was invited to create the system for an autonomous underwater drone for the lab.

% I started scuba diving at the age of 11. By 15 I had over 100 dives under my belt and was trained as a Rescue Scuba Diver. At 19 I became a certified Scuba instructor. During my undergrad at the University of Michigan, I was one of the leaders of the Human Powered Submarine Team. We designed, built, and raced a submarine powered by a scuba diver inside pedaling bicycle pedals. After school, I volunteered as a Scuba diver at the Shedd Aquarium in Chicago. I proposed to my wife underwater. This is all to say: I have been personally interested in Scuba diving and reef conservation for a very long time. This made me the perfect candidate to leverage my skills in robotics to design a vehicle to help divers get more done.

% Beyond the robotic boat mentioned above, I have a history of building platforms and technology that enable others to achieve more. My first job out of college was at a data center where I developed a novel containerized data center product. This allowed for a drastic decrease in the construction time of data centers while increasing energy efficiency and reliability. Later, I founded several startups with the aim of creating tools and experiences to help. Mm y last and most successful startup created a special type of event that made it easier for strangers to meet and make friends. And immediately before joining ASU for my Master’s, I built a platform for a mortgage company that gave more transparency and power to borrowers in the loan process. This is all to say I am very passionate about building platforms and systems that enable innovation. This, along with the diving, drew me to the challenge of developing the uDrone. 

\section{Challenges of Underwater}
Working in underwater environments presents a number of challenges not experienced by aerial, ground, or surface vehicles. These include physical challenges, difficulty of communication, and lack of global positioning systems. 
\subsection{Physical Challanges}
First and foremost is needing to seal the vehicle for water and pressure. The uDrone will need to maintain a watertight seal down to 30m of water. This is made more challenging by the need to have some components exterior to the enclosure which houses the electronics, necessitating the creation of watertight pass-throughs. As the vehicle goes to deeper depths this becomes increasingly harder. Every 10m of water adds an additional atmosphere of pressure to the vehicle. This pressure causes two problems. First, enclosures must be strong enough to resist the force of the pressure change and not fail structurally. Second, o-rings and other seals shrink under pressure, which can lead to leaks. 

In order to achieve this, the uDrone uses a Blue Robotics enclosure to keep the electronics dry inside. In order to handle the cable pass-throughs, and o-ring sealed passthrough is attached to the rear of the enclosure and then the potting compound is used to seal the area between the cable and passthrough. 

An underwater vehicle also has to manage its buoyancy. If the vehicle is too heavy or too light it will end up moving through the water column at an angle, increasing drag and decreasing efficiency. Additionally, if the centers of buoyancy and mass to not line up then there will be a moment acting on the vehicle, complicating modeling and control. While this presents challenges, it can also lead to opportunities. By creating a vehicle that is slightly positively buoyant or by having a mechanism that allows a vehicle to drop ballast, a vehicle will float and can be recovered in the event of failure. Since the main camera on the uDrone is front-facing, the center of buoyancy can be set slightly behind the center of gravity. This will cause it to point down when hovering, but in motion, the thrusters can compensate for the moment. 

Many autonomous vehicles are able to simplify certain aspects of the model. For example, a vehicle moving through the air does not need to worry about added mass from the layer of air near the vehicle surface. In fact, air drag is often ignored unless a drone is moving at high speeds. This is not possible underwater due to the greater viscosity and density of water compared to air. In order to get reliable system modeling, expensive and complicated fluid dynamic software must be used. Since this is often impractical for applications such as the uDrone, the values for added mass and drag are often estimated and then verified experimentally. 
Similar to the effects of wind on an aerial vehicle, underwater vehicles must be able to compensate for currents. This is made more challenging as the speed of the vehicle cannot be verified by GPS. Instead, visual odometry must be used to determine the effects of current on the vehicle and aid in compensation. 

\subsection{Communication Challanges}
Wifi or radio signals do not penetrate to useful depths underwater. Radio waves in the MHz or higher frequency range typically penetrate less than 1 meter underwater (signals citation). Since the radio controllers used on drones communicate at 900 MHz and wifi uses 2.4 GHz or 5 Ghz, none of these are viable communication options. 
Optimal communication is possible but is limited by line of sight and water column turbidity. Optimal communication is has been implemented in a simulation for the uDrone, where it follows a boat that has a visual marker mounted underneath (Cite Harish). This is reasonable for some use cases of the uDrone, but cannot be relied on for all scenarios.
https://www.hydromea.com/underwater-wireless-communication/
The most common method of real-time communication with underwater vehicles is via a tether. This allows for fast and reliable connections but limits the movement of the vehicle based on cable length and cable management. While the uDrone is capable of functioning with a cable for testing purposes, it is designed to operate without one. 
The only reliable method of wireless underwater communication is via acoustic transponders. However, these devices tend to be large, expensive, and power-hungry. While one day it might be useful to have wireless acoustic communications at specific sites of uDrone deployment, it is not practical in the initial use case.

\subsection{Global Positioning & Navigation Challenges}
GPS satellites transmit data at high frequencies, in the GHz range. For this reason, traditional, satellite-based GPS is not practical for use underwater. The uDrone is still outfitted with GPS so it can determine its location on the surface, but it cannot rely on this as a method of localization while underwater.
It is possible to use acoustic signals to localize an underwater vehicle. This requires three or more surface-based acoustic broadcasters. An underwater vehicle can then triangulate between these sources in order to determine its position relative to the sources. This is basically the equivalent to creating an underwater GPS system. While it is possible that this could be implemented at some operational sites in the future, it is not practical to build in the beginning and is therefore not a reliable way for the uDrone to localize.
Due to the lack of GPS, the uDrone must use visual and inertial based odometry to determine its location. LIDAR, which is the measurement tool of choice of most aerial and land-based robots, is severely limited underwater for two reasons. First, lower frequency light is absorbed quickly underwater, and since most lidars use these lower frequencies, the range is severely limited. Second, murky water greatly limits the ability of light to pass through water, and since it is very common to have many dissolved particles in the water column this makes LIDAR inaccurate. Therefore, the best type of sensor to measure distance underwater is with an echosounder. 
Another tool typically used by land-based robotics to understand their surroundings is 3D or RGB-D cameras. There are two main types of RGB-D cameras: stereo and infrared. Stereo cameras calculate the pixel depth of an image by looking at the difference in location between the images from two cameras placed a known distance apart. Infrared cameras work by projecting a pattern in infrared light onto surfaces and using the deformations to calculate distance. Since lower frequency lights, such as red and infrared, are absorbed most quickly by water, infrared RGB-D cameras do not work well underwater. For this reason, we selected a stereo camera to handle the visual needs of the uDrone. 
